After comparing all models, we understand that the best one is the Voting Classifier.
This makes sense given that all previous models performed quite well, so this additional step (selecting among all of them) was a likely candidate to give similar quality or better results.

\subsection{PCA}
Principal Component Analysis (PCA) is a technique for reducing the dimensionality of data by finding the components that best fit it, then leaving aside components that are not significant.\\
Applying PCA on the training data, we detect that over 99.82\% of the signal can be represented by only one dimension (and a second one holds the following 0.14\% of it).
Given the initial 21 dimensions of the data, we might expect that taking only 2 dimensions representing 99.96\% of the signal would improve the results for any of the models.
Even more, 3 dimensions hold 99.99\% of the signal, so this is the number we choose.

\subsubsection{Model Evaluation}
Trying to make profit of the dimensionality reduction, we evaluate the Voting Classifier applied to the updated dataset.
\begin{itemize}
    \item Mean Average Error (train): 0.2835790729904527
    \item Mean Average Error (validation): 0.3135201724668925
\end{itemize}
These numbers show that applying PCA on the dataset for reducing the number of features from 21 to 3 decreases the performance of the classifier down to ~70\%.
